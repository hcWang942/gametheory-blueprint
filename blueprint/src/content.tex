\title{Auction Theory Blueprint} 


\home{https://hcWang942.github.io/gametheory-blueprint/}
\github{https://github.com/hcWang942/gametheory-blueprint/}
\dochome{https://hcWang942.github.io/gametheory-blueprint/doc/}

% \home{localhost:8000}
% \dochome{localhost:8000/doc}

\maketitle


\tableofcontents
\section{Introduction}



\section{Example}
Here you can use LaTex. \cite{marcus}. You must cite something to make it work.

\begin{lemma}\label{two_different_elements}
    \leanok
    \lean{two_different_elements}
    Here is a documentation.
\end
\begin{proof}
    Here is a proof.
\end{proof}

\begin{definition}\label{Auction}
    \leanok
    \lean{Auction}
    \uses{two_different_elements, }
    Here is a definition.
\end{definition}

\begin{definition}\label{maxb}
    \leanok
    \lean{maxb}
    Here is a definition.
\end{definition}

\begin{lemma}\label{exists_max}
    \leanok
    \lean{exists_max}
    Here is a definition.
\end{lemma}
\begin{proof}
    Here is a proof.
\end{proof}


\begin{definition}\label{winner}
    \leanok
    \lean{winner}
    \uses{exists_max }
    Here is a definition.
\end{definition}

\begin{lemma}\label{winner_take_max}
    \leanok
    \lean{winner_take_max}
    \uses{maxb, exists_max, winner}
    Here is a definition.
\end{lemma}
\begin{proof}
    Here is a proof.
\end{proof}

\begin{lemma}\label{delete_i_nonempty}
    \leanok
    \lean{delete_i_nonempty}
    Here is a definition.
\end{lemma}
\begin{proof}
    Here is a proof.
\end{proof}

\begin{definition}\label{B}
    \leanok
    \lean{B}
    \uses{delete_i_nonempty}
    Here is a definition.
\end{definition}

\begin{definition}\label{secondprice}
    \leanok
    \lean{secondprice}
    \uses{winner, B}
    Here is a definition.
\end{definition}

\begin{definition}\label{utility}
    \leanok
    \lean{utility}
    \uses{winner, secondprice}
    Here is a definition.
\end{definition}

\begin{lemma}\label{utility_winner}
    \leanok
    \lean{utility_winner}
    \uses{utility, secondprice}
    Here is a definition.
\end{lemma}
\begin{proof}
    Here is a proof.
\end{proof}

\begin{lemma}\label{utility_loser}
    \leanok
    \lean{utility_loser}
    \uses{utility, winner}
    Here is a definition.
\end{lemma}
\begin{proof}
    Here is a proof.
\end{proof}

\begin{definition}\label{dominant}
    \leanok
    \lean{dominant}
    \uses{utility}
    Here is a definition.
\end{definition}


\begin{lemma}\label{gt_wins}
    \leanok
    \lean{gt_wins}
    \uses{winner, maxb, winner_take_max}
    Here is a definition.
\end{lemma}
\begin{proof}
    Here is a proof.
\end{proof}



\begin{lemma}\label{b_winner_max}
    \leanok
    \lean{b_winner_max}
    \uses{winner, maxb, winner_take_max}
    Here is a definition.
\end{lemma}
\begin{proof}
    Here is a proof.
\end{proof}

\begin{lemma}\label{b_winner}
    \leanok
    \lean{b_winner}
    \uses{winner, secondprice ,winner_take_max,  delete_i_nonempty}
    Here is a definition.
\end{lemma}
\begin{proof}
    Here is a proof.
\end{proof}

\begin{lemma}\label{b_loser_max}
    \leanok
    \lean{b_loser_max}
    \uses{winner, maxb ,winner_take_max}
    Here is a definition.
\end{lemma}
\begin{proof}
    Here is a proof.
\end{proof}

\begin{lemma}\label{utility_nneg}
    \leanok
    \lean{utility_nneg}
    \uses{utility, winner, secondprice ,winner_take_max, maxb}
    Here is a definition.
\end{lemma}
\begin{proof}
    Here is a proof.
\end{proof}

\begin{lemma}\label{valuation_is_dominant}
    \leanok
    \lean{valuation_is_dominant}
    \uses{dominant, winner, gt_wins, utility_winner, secondprice , 
          delete_i_nonempty, utility_nneg}
    Here is a definition.
\end{lemma}
\begin{proof}
    Here is a proof.
\end{proof}

\begin{definition}\label{Utility.FirstPrice}
    \leanok
    \lean{Utility.FirstPrice}
    \uses{winner}
    Here is a definition.
\end{definition}

\begin{lemma}\label{utility_first_price_winner}
    \leanok
    \lean{utility_first_price_winner}
    \uses{winner, Utility.FirstPrice}
    Here is a definition.
\end{lemma}
\begin{proof}
    Here is a proof.
\end{proof}

\begin{lemma}\label{utility_first_price_loser}
    \leanok
    \lean{utility_first_price_winner}
    \uses{winner, Utility.FirstPrice}
    Here is a definition.
\end{lemma}
\begin{proof}
    Here is a proof.
\end{proof}

\begin{definition}\label{Dominant.FirstPrice}
    \leanok
    \lean{Dominant.FirstPrice}
    \uses{Utility.FirstPrice}
    Here is a definition.
\end{definition}

\begin{theorem}\label{first_price_has_no_dominant_strategy}
    \leanok
    \lean{first_price_has_no_dominant_strategy}
    \uses{Dominant.FirstPrice, winner, gt_wins}
    Here is a definition.
\end{lemma}
\begin{proof}
    Here is a proof.
\end{proof}

