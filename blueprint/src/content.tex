\title{Auction Theory Blueprint} 


\home{https://hcWang942.github.io/gametheory-blueprint/}
\github{https://github.com/hcWang942/gametheory-blueprint/}
\dochome{https://hcWang942.github.io/gametheory-blueprint/doc/}

% \home{localhost:8000}
% \dochome{localhost:8000/doc}

\maketitle


\tableofcontents
\section{Introduction}
\section{Introduction}

This section formalizes core concepts and results in auction theory using the Lean theorem prover. Our approach focuses on establishing a rigorous framework that captures the dynamics and strategic behaviors in auctions.

\subsection{Main Definitions}

\begin{itemize}
    \item \textbf{Auction}: Defines an auction setup including bidders and their respective valuations.
    \item \textbf{maxb}: A function to compute the highest bid within a given set of bids.
    \item \textbf{winner}: Identifies the auction winner as the bidder with the highest bid.
    \item \textbf{utility}: Calculates the utility for each bidder, taking into account their bid and the auction outcome.
    \item \textbf{dominant}: Determines whether a bidding strategy is dominant for a bidder.
\end{itemize}

\subsection{Main Results}

\begin{itemize}
    \item \textbf{exists\_max}: Proves the existence of a participant whose bid is equal to the highest bid.
    \item \textbf{winner\_take\_max}: Ensures that the bid of the winner is the highest among all bidders.
    \item \textbf{b\_winner}: Asserts that the winner's bid surpasses or matches the second highest bid.
    \item \textbf{valuation\_is\_dominant}: Demonstrates that bidding one's own valuation is a dominant strategy.
\end{itemize}

\subsection{Notations}

\begin{itemize}
    \item \( |b| \): Denotes a bidding function.
    \item \( \text{maxb}(b) \): Represents the highest bid in the function \( b \).
    \item \( B_i \): Stands for the maximal bid of all participants except participant \( i \).
\end{itemize}

\subsection{Implementation Notes}

The defined structures and functions are crafted to accommodate multiple bidders, enhancing the realism and applicability of auction models. Key definitions such as \textit{winner} and \textit{maxb} leverage Lean's \textit{Finset} and \textit{Classical} logic modules to effectively manage non-constructive scenarios.

\subsection{References}

Our theoretical groundwork is supported by classical texts and research papers in auction theory, adapting their general proofs to the structured environment of Lean.

\subsection{Tags}

Keywords relevant to this work include \textit{auction}, \textit{game theory}, \textit{economics}, \textit{bidding}, and \textit{valuation}.



\section{Example}
Here you can use LaTex. \cite{marcus}. You must cite something to make it work.

\begin{lemma}\label{two_different_elements}
    \leanok
    \lean{two_different_elements}
    Here is a documentation.
\end{lemma}

\begin{proof}
    Here is a proof.
\end{proof}

\begin{definition}\label{Auction}
    \leanok
    \lean{Auction}
    \uses{two_different_elements}
    This is the structure definition of the formalized auction theory project.
\end{definition}

\begin{definition}\label{maxb}
    \leanok
    \lean{maxb}
    maxb is set to compute the highest bid given a bidding function `b`.
\end{definition}

\begin{lemma}\label{exists_max}
    \leanok
    \lean{exists_max}
    There exists a participant `i` whose bid equals the highest bid.
\end{lemma}
\begin{proof}
    Here is a proof.
\end{proof}


\begin{definition}\label{winner}
    \leanok
    \lean{winner}
    \uses{exists_max }
    winner: The participant with the highest bid.
\end{definition}

\begin{lemma}\label{winner_take_max}
    \leanok
    \lean{winner_take_max}
    \uses{maxb, exists_max, winner}
    Here is a definition.
\end{lemma}
\begin{proof}
    The bid of the winner equals the highest bid.
\end{proof}

\begin{lemma}\label{delete_i_nonempty}
    \leanok
    \lean{delete_i_nonempty}
    Removing a participant `i` from all participants still leaves a non-empty set.
\end{lemma}
\begin{proof}
    Here is a proof.
\end{proof}

\begin{definition}\label{B}
    \leanok
    \lean{B}
    \uses{delete_i_nonempty}
    State that `B i` is the maximal bid of all participants but `i`.
\end{definition}

\begin{definition}\label{secondprice}
    \leanok
    \lean{secondprice}
    \uses{winner, B}
    The secondprice refers to the second highest bid, i.e. the highest bid excluding the winner’s bid.
\end{definition}

\begin{definition}\label{utility}
    \leanok
    \lean{utility}
    \uses{winner, secondprice}
    Defines the utility of participant `i`, which is their valuation minus the second highest bid if `i` is the winner, otherwise, it's 0.
\end{definition}

\begin{lemma}\label{utility_winner}
    \leanok
    \lean{utility_winner}
    \uses{utility, secondprice}
    If `i` is the winner, then their utility is their valuation minus the second highest bid.
\end{lemma}
\begin{proof}
    Here is a proof.
\end{proof}

\begin{lemma}\label{utility_loser}
    \leanok
    \lean{utility_loser}
    \uses{utility, winner}
    If `i` is not the winner, then their utility is 0.
\end{lemma}
\begin{proof}
    Here is a proof.
\end{proof}

\begin{definition}\label{dominant}
    \leanok
    \lean{dominant}
    \uses{utility}
    A strategy is dominant if bidding `bi` ensures that bidder `i`'s utility is maximized relative to any other bids `b'` where `b i = bi`.
\end{definition}


\begin{lemma}\label{gt_wins}
    \leanok
    \lean{gt_wins}
    \uses{winner, maxb, winner_take_max}
    Here is a definition.
\end{lemma}
\begin{proof}
    If `i`'s bid is higher than all other bids, then `i` is the winner.
\end{proof}



\begin{lemma}\label{b_winner_max}
    \leanok
    \lean{b_winner_max}
    \uses{winner, maxb, winner_take_max}
    The bid of the winner is always greater than or equal to the bids of all other participants.
\end{lemma}
\begin{proof}
    Here is a proof.
\end{proof}

\begin{lemma}\label{b_winner}
    \leanok
    \lean{b_winner}
    \uses{winner, secondprice ,winner_take_max,  delete_i_nonempty}
    The bid of the winner is always greater than or equal to the second highest bid.
\end{lemma}
\begin{proof}
    Here is a proof.
\end{proof}

\begin{lemma}\label{b_loser_max}
    \leanok
    \lean{b_loser_max}
    \uses{winner, maxb ,winner_take_max}
    If `i` is not the winner, then the highest bid excluding `i` is equal to the overall highest bid.
\end{lemma}
\begin{proof}
    Here is a proof.
\end{proof}

\begin{lemma}\label{utility_nneg}
    \leanok
    \lean{utility_nneg}
    \uses{utility, winner, secondprice ,winner_take_max, maxb}
    Here is a definition.
\end{lemma}
\begin{proof}
    Utility is non-negative if the bid equals the valuation.
\end{proof}

\begin{lemma}\label{valuation_is_dominant}
    \leanok
    \lean{valuation_is_dominant}
    \uses{dominant, winner, gt_wins, utility_winner, secondprice , 
          delete_i_nonempty, utility_nneg}
          Proves that the strategy of bidding one's valuation is a dominant strategy for `i`.
\end{lemma}
\begin{proof}
    Here is a proof.
\end{proof}

\begin{definition}\label{Utility.FirstPrice}
    \leanok
    \lean{Utility.FirstPrice}
    \uses{winner}
    Computes the utility for a first price auction where the winner pays their bid.
\end{definition}

\begin{lemma}\label{utility_first_price_winner}
    \leanok
    \lean{utility_first_price_winner}
    \uses{winner, Utility.FirstPrice}
    If `i` is the winner in a first price auction, their utility is their valuation minus their bid.
\end{lemma}
\begin{proof}
    Here is a proof.
\end{proof}

\begin{lemma}\label{utility_first_price_loser}
    \leanok
    \lean{utility_first_price_winner}
    \uses{winner, Utility.FirstPrice}
    If `i` is not the winner in a first price auction, their utility is 0.
\end{lemma}
\begin{proof}
    Here is a proof.
\end{proof}

\begin{definition}\label{Dominant.FirstPrice}
    \leanok
    \lean{Dominant.FirstPrice}
    \uses{Utility.FirstPrice}
    Defines a dominant strategy in the context of a first price auction.
\end{definition}

\begin{theorem}\label{first_price_has_no_dominant_strategy}
    \leanok
    \lean{first_price_has_no_dominant_strategy}
    \uses{Dominant.FirstPrice, winner, gt_wins}
    There is no dominant strategy in a first price auction for any `i` and any bid `bi`.
\end{theorem}
\begin{proof}
    Here is a proof.
\end{proof}







% \begin{definition}\label{SingleParameterEnvironment}
%     \leanok
%     \lean{SingleParameterEnvironment}
%     Here is a definition.
% \end{definition}

% \begin{definition}\label{Nonempty E.I}
%     \leanok
%     \lean{Nonempty E.I}
%     Here is a definition.
% \end{definition}

% \begin{definition}\label{Fintype E.I }
%     \leanok
%     \lean{Dominant.FirstPrice}
%     Here is a definition.
% \end{definition}

% \begin{definition}\label{Nonempty (E.feasibleSet) }
%     \leanok
%     \lean{Dominant.FirstPrice}
%     Here is a definition.
% \end{definition}

% \begin{definition}\label{CoeFun E.feasibleSet}
%     \leanok
%     \lean{CoeFun E.feasibleSet}
%     Here is a definition.
% \end{definition}


% \begin{definition}\label{DirectRevelationMechanism}
%     \leanok
%     \lean{Dominant.FirstPrice}
%     \uses{Utility.FirstPrice}
%     Here is a definition.
% \end{definition}


% \begin{definition}\label{utility}
%     \leanok
%     \lean{utility}
%     Here is a definition.
% \end{definition}

% \begin{definition}\label{dominant}
%     \leanok
%     \lean{utility}
%     \uses{utility}
%     Here is a definition.
% \end{definition}

% \begin{definition}\label{dsic}
%     \leanok
%     \lean{dsic}
%     \uses{dominant}
%     Here is a definition.
% \end{definition}


% \begin{definition}\label{nondecreasingy}
%     \leanok
%     \lean{nondecreasing}
%     Here is a definition.
% \end{definition}


% \begin{definition}\label{with_hole}
%     \leanok
%     \lean{with_hole}
%     Here is a definition.
% \end{definition}

% \begin{lemma}\label{filled_hole_retrieve}
%     \leanok
%     \lean{filled_hole_retrieve}
%     \uses{with_hole}
%     Here is a definition.
% \end{lemma}
% \begin{proof}
%     Here is a proof.
% \end{proof}


% \begin{lemma}\label{filled_hole_retrieve_other}
%     \leanok
%     \lean{utility_nneg}
%     \uses{with_hole}
%     Here is a definition.
% \end{lemma}
% \begin{proof}
%     Here is a proof.
% \end{proof}


% \begin{lemma}\label{filled_hole_almost_equal}
%     \leanok
%     \lean{filled_hole_almost_equal}
%     \uses{with_hole, filled_hole_retrieve_other}
%     Here is a definition.
% \end{lemma}
% \begin{proof}
%     Here is a proof.
% \end{proof}


% \begin{lemma}\label{almost_equal_fill_hole}
%     \leanok
%     \lean{almost_equal_fill_hole}
%     \uses{with_hole}
%     Here is a definition.
% \end{lemma}
% \begin{proof}
%     Here is a proof.
% \end{proof}

% \begin{lemma}\label{filled_hole_replace}
%     \leanok
%     \lean{filled_hole_replace}
%     \uses{with_hole}
%     Here is a definition.
% \end{lemma}
% \begin{proof}
%     Here is a proof.
% \end{proof}

% \begin{lemma}\label{unfill_fill_hole}
%     \leanok
%     \lean{unfill_fill_hole}
%     \uses{with_hole}
%     Here is a definition.
% \end{lemma}
% \begin{proof}
%     Here is a proof.
% \end{proof}

% \begin{definition}\label{monotone}
%     \leanok
%     \lean{monotone}
%     Here is a definition.
% \end{definition}

% \begin{definition}\label{implementable}
%     \leanok
%     \lean{implementable}
%     Here is a definition.
% \end{definition}

% \begin{theorem}\label{payment_sandwich}
%     \leanok
%     \lean{payment_sandwich}
%     \uses{with_hole, filled_hole_almost_equal, utility}
%     Here is a definition.
% \end{lemma}
% \begin{proof}
%     Here is a proof.
% \end{proof}

% \begin{theorem}\label{implementable_impl_monotone}
%     \leanok
%     \lean{implementable_impl_monotone}
%     \uses{implementable, monotone, payment_sandwich, 
%     nonpos_of_mul_nonpos_righ, with_hole}
%     Here is a definition.
% \end{lemma}
% \begin{proof}
%     Here is a proof.
% \end{proof}


% \begin{definition}\label{magic_payment_rule}
%     \leanok
%     \lean{magic_payment_rule}
%     \uses{with_hole}
%     Here is a definition.
% \end{definition}


% \begin{definition}\label{with_magic}
%     \leanok
%     \lean{with_hole}
%     \uses{magic_payment_rule}
%     Here is a definition.
% \end{definition}

% \begin{definition}\label{utility_exp}
%     \leanok
%     \lean{utility_exp}
%     \uses{utility, with_magic, with_hole}
%     Here is a definition.
% \end{definition}









